\subsection{Уравнение неразрывности. Уравнение Эйлера и Навье-Стокса. Массовые и поверхностные силы.}
Массовые силы $F_\text{масс}$ пропорциональны $dm=\rho dV$.
К ним относятся: силы притяжения (Земли, Луны, Солнца и т.д.) и силы инерции (Кориолиса, центробежная).
Приложены к центру масс элементарного объёма $dV$.

Поверхностные силы $F_\text{поверх}=\sum_{j}\frac{\partial\tau_{ij}}{\partial x_{j}}dV$, где $\tau_{ij}$ -- тензор напряжений.
Приложены к граням элементарного объёма $dV$.

Исходя из второго закона Ньютона $m\vec{a}=\sum\vec{F}_\text{масс}+\sum\vec{F}_\text{поверх}$ получаем уравнение движения вязкой жидкости (уравнение Навье-Стокса) \cite{Nosov2019-6}:
\begin{align}\label{eq-2-2-1}
\begin{split}
\frac{d\vec{v}}{dt}
&=
\frac{\partial\vec{v}}{\partial t}+\left(\vec{v}\cdot\vec{\nabla}\right)\vec{v}
\\
&=
\underbrace{\vec{g}+2\left[\vec{v}\times\vec{\Omega}\right]}_\text{массовые силы}
+\underbrace{\frac{1}{\rho}\sum_{j}\frac{\partial\tau_{ij}}{\partial x_{j}}}_\text{поверхностные силы}
\\
&=
\underbrace{\vec{g}}_\text{сила тяжести}
+\underbrace{2\left[\vec{v}\times\vec{\Omega}\right]}_\text{сила Кориолиса}
-\underbrace{\frac{\vec{\nabla}p}{\rho}}_\text{сила градиента давления}
\\
&+
\underbrace{\nu\Delta\vec{v}+\left(\zeta+\frac{\nu}{3}\right)\vec{\nabla}\left(\vec{\nabla}\cdot\vec{v}\right)}_\text{сила вязкого трения}
\end{split}
\end{align}
где $\nu$ -- коэффициент кинематической вязкости, а коэффициент $\zeta$ часто называют второй вязкостью.

В случае отсутствия вязкости уравнение Навье-Стокса переходит в уравнение Эйлера (уравнение движения идеальной жидкости):
\begin{equation}\label{eq-2-2-2}
\frac{\partial\vec{v}}{\partial t}+\left(\vec{v}\cdot\vec{\nabla}\right)\vec{v}=\vec{g}+2\left[\vec{v}\times\vec{\Omega}\right]-\frac{\vec{\nabla}p}{\rho}
\end{equation}

Векторное уравнение движения имеет 5 неизвестных величин (три компоненты скорости $\vec{v}$, давление $p$ и плотность $\rho$), поэтому для его замыкания используются уравнение состояния $\rho=\rho(p)$ и уравнение неразрывности (закон сохранения массы) для сплошной среды:
\begin{equation}\label{eq-2-2-3}
\frac{\partial\rho}{\partial t}+\vec{\nabla}\cdot(\rho\vec{v})=0
\end{equation}

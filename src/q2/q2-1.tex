\subsection{Понятие сплошной среды. Подходы Лагранжа и Эйлера к описанию движения сплошной среды.}
Основные гипотезы механики сплошных сред (МСС) \cite{Gorbachev2018-1}:
\begin{enumerate}
\item Гипотеза сплошности. Тело называется сплошным, если в любом бесконечно малом объёме содержится бесконечное количество материальных точек. Эта гипотеза позволяет перейти от уравнений для отдельной материальной точки к уравнениям для локальных характеристик.
\item Движение сплошной среды происходит в евклидовом пространстве и в абсолютном времени.
\end{enumerate}

Движение сплошной среды можно изучать, исходя из двух подходов:
\begin{enumerate}
\item Подход Лагранжа. Фиксируем материальную точку и наблюдаем за ней в пространстве и времени.
\item Подход Эйлера. Фиксируем точку пространства и наблюдаем за ней во времени.
\end{enumerate}

Объектами исследования МСС являются: жидкие, газообразные и твёрдые деформируемые тела.
Подход Лагранжа применяется для описания движения твёрдых деформируемых тел, а Эйлера -- жидкостей и газов.

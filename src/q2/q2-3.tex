\subsection{Подходы к упрощению уравнений гидродинамики. Потенциальное и вихревое движение. Гидростатика. Уравнение Бернулли.}
Гидродинамические приближения \cite{Nosov2019-6}:
\begin{enumerate}
\item Среда обладает постоянной плотностью, т.е. $\rho=\rho_0=const$.
\item Течение стационарно, т.е. $\frac{\partial}{\partial t}=0$.
\item Жидкость идеальная (невязкая), т.е. пренебрежение силой вязкого трения.
\item Линейное приближение (в случае малости скорости), т.е. пренебрежение членом $\left(\vec{v}\cdot\vec{\nabla}\right)\vec{v}$ в (\ref{eq-2-2-1}).
\end{enumerate}
Геофизические приближения \cite{Nosov2019-6}:
\begin{enumerate}
\item Гидростатическое приближение (по вертикали):
\begin{equation}\label{eq-2-3-1}
-\frac{\vec{\nabla}p}{\rho}+\vec{g}=0, \vec{\nabla}=\vec{k}\frac{\partial}{\partial z}
\end{equation}
\item Геострофическое приближение (по горизонтали):
\begin{equation}\label{eq-2-3-2}
-\frac{\vec{\nabla}p}{\rho}+2\left[\vec{v}\times\vec{\Omega}\right]=0, \vec{\nabla}=\vec{i}\frac{\partial}{\partial x}+\vec{j}\frac{\partial}{\partial y}
\end{equation}
\end{enumerate}

Все движения жидкостей подразделяются на потенциальные и вихревые.
Если относительная завихрeнность $\vec{\xi}=\left[\vec{\nabla}\times\vec{v}\right]=0$, то движение является потенциальным, в противном случае -- вихревым.

Рассмотрим стационарное течение идеальной жидкости в поле силы тяжести, но без действия силы Кориолиса.
С учётом соотношения Максвелла $dH=Tds+Vdp=\frac{dp}{\rho}$ (где $H \left[\frac{\text{Дж}}{\text{кг}}\right]$ -- энтальпия) и формулой из векторного анализа $\frac{1}{2}\vec{\nabla}v^2=\left[\vec{v}\times\left[\vec{\nabla}\times\vec{v}\right]\right]+\left(\vec{v}\cdot\vec{\nabla}\right)\vec{v}$ уравнение Эйлера (\ref{eq-2-2-2}) переходит в уравнение:
\begin{equation}\label{eq-2-3-3}
\frac{1}{2}\vec{\nabla}v^2 -\left[\vec{v}\times\left[\vec{\nabla}\times\vec{v}\right]\right]=\vec{g}-\vec{\nabla}H
\end{equation}

Проецируя уравнение (\ref{eq-2-3-3}) на направление вдоль линии тока жидкости $l$ (т.е. рассматривая потенциальное движение), получаем уравнение Бернулли \cite{Ландау-Лившиц-1986}:
\begin{equation}\label{eq-2-3-4}
\frac{\partial}{\partial l}\left(\frac{v^2}{2}+H+gz\right)=0\Rightarrow\frac{v^2}{2}+H+gz=const
\end{equation}

\subsection{Физические свойства воздуха. Влажность воздуха. Уравнение состояние воздуха. Баротропность и бароклинность.}
К физическим свойствам воздуха можно отнести: температуру, давление и влажность.

Степень влажности воздуха определяется тремя показателями: абсолютной, максимальной и относительной влажностью.
Абсолютная влажность [$\frac{\text{гр}}{\text{м}^{3}}$] -- количество водяных паров воздуха при данной температуре.
Максимальная влажность [$\frac{\text{гр}}{\text{м}^{3}}$] -- максимальное количество водяных паров воздуха при данной температуре.
Относительная влажность [\%] –- отношение абсолютной влажности к максимальной.

Бароклинная среда -- это среда, уравнение состояния которого зависит не только от давление, но и от других факторов, т.е. $\rho=\rho(p, T, ...)$.

Баротропная среда -- это среда, уравнение состояния которого определяется только давлением, т.е. $\rho=\rho(p)\Leftrightarrow\left[\vec{\nabla}\rho\times\vec{\nabla}p\right]=0$.

Уравнение состояния \textit{сухого} воздуха определяется из уравнения Менделеева-Клапейрона:
\begin{equation}\label{eq-1-2-1}
pV=\frac{m}{\mu}RT\Rightarrow \rho=\frac{p}{R_{\mu}T}
\end{equation}
где $R_{\mu}=\frac{R}{\mu}$ -- значение газовой постоянной для атмосферы Земли.

Если к (\ref{eq-1-2-1}) добавить составляющую, связанную с водяным паром, то получим уравнение состояния \textit{влажного} воздуха \cite{Nosov2019-5}:
\begin{align}\label{eq-1-2-2}
\begin{split}
\rho
&=\frac{p-e}{R_{\mu}T}+\frac{e}{R_{w}T}
\\
&=\frac{p}{R_{\mu}T}\left(1-\left[1-\frac{R_{\mu}}{R_{w}}\right]\frac{e}{p}\right)
\\
&\approx\frac{p}{R_{\mu}T}\left(1-0.38\frac{e}{p}\right)
\end{split}
\end{align}
где $e$ -- парциальное давление водяного пара.

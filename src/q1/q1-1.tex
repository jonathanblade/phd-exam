\subsection{Происхождение атмосферы и гидросферы Земли. Условия существования атмосферы и гидросферы.}
Существует две основных гипотезы происхождения атмосферы и гидросферы Земли \cite{Nosov2019-3}:
\begin{enumerate}
\item Атмосфера и гидросфера Земли сформировались в результате дегазации мантии (\textbf{ведущая гипотеза}).
\item Атмосфера и гидросфера Земли были принесены извне посредством вещества комет, метеоритов и т.п.
\end{enumerate}
Условиями существования атмосферы являются \cite{Nosov2019-3}:
\begin{enumerate}
\item $v_{T}<v_{2}$, где $v_{T}$ -- скорость теплового движения молекул, а $v_{2}=\sqrt{\frac{2GM_{\oplus}}{R_{\oplus}}}\approx\sqrt{2gR_{\oplus}}\approx\text{11.2 км/с}$ -- вторая космическая скорость. Эффект диссипации\footnote{Диссипация атмосфер -- ускользания газов из атмосфер космических тел, вызванное тепловым движением атомов и молекул.} атмосфер присутствует всегда. Это следует из распределения Максвелла, где имеется участок функции плотности вероятности тепловой скорости $f(v)$ при $v>v_{2}$. Такие газы, как водород и гелий, довольно эффективно\footnote{Время \textquote{полной} диссипации из атмосферы Земли для водорода составляет несколько лет, а для гелия -- несколько млн. лет.} диссипируют из атмосферы Земли.
\item Наличие дополнительных источников газов. \textquote{Лёгкие} газы постоянно восполняется  при помощи дополнительных источников (например, той же дегазации мантии).
\item Наличие стратификации. Длина свободного пробега \textquote{тяжёлых} газов не позволяет им спокойно покинуть атмосферу Земли.
\end{enumerate}
Условиями существования гидросферы (океана) являются \cite{Nosov2019-3}:
\begin{enumerate}
\item Температура на планете должна быть выше температуры плавления вещества океана.
\item Парциальное давление газообразной фазы вещества океана должно быть выше насыщающего давления. Иначе говоря, вещества океана на планете должно быть достаточно много.
\item Температура и давление вещества океана должны быть ниже критической\footnote{Критическая точка -- состояние вещества, при котором его жидкая и газообразная фазы становятся не различимыми. Для воды это 374 \textcelsius, 218 атм.} точки.
\end{enumerate}

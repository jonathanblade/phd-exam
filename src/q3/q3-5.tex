\subsection{Геострофическое приближение. Число Россби. Геострофическое приспособление. Радиус деформации Россби. }
Геострофическое приспособление -- процесс восстановления геострофического ветра\footnote{Геострофический ветер -- ветер вдоль линий изобар.} при его возмущения (т.е. стремление сплошной среды достичь геострофического равновесия).

Вклад эффекта вращения Земли в процесс движения характеризуется числом Россби \cite{Nosov2019-6}:
\begin{equation}
R_0=\frac{\left|\frac{d\vec{v}}{dt}\right|}{\left|2\left[\vec{v}\times\vec{\Omega}\right]\right|}\sim\frac{\frac{|v_{xy}|}{\tau}}{f|v_{xy}|}=\frac{|v_{xy}|}{fL}
\end{equation}
где $\tau=\frac{L}{|v_{xy}|}$ -- временной масштаб движения по горизонтали, $f$ -- параметр Кориолиса.
Если число Россби $R_0\ll 1$\footnote{На экваторе число Россби $R_0\rightarrow\infty$, т.е. геострофическое приближение не работает.}, то для такого течения эффект вращения Земли значителен и справедливо геострофическое приближение (\ref{eq-2-3-2}).

Радиус деформации Россби -- горизонтальный масштаб, при котором сила Кориолиса (эффект вращения Земли) уравновешивается силой градиента атмосферного давления (эффектом плавучести) \cite{Белоненко-Новоселова-2019}.
Иначе говоря, радиус деформации Россби описывает характерный горизонтальный масштаб атмосферного возмущения (геострофического приспособления).

Баротропный радиус деформации Россби:
\begin{equation}
R_1=\frac{\sqrt{gH}}{f}
\end{equation}

Бароклинный радиус деформации Россби:
\begin{equation}
R_2=\frac{NH}{f}
\end{equation}

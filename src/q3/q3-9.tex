\subsection{Волновые движение в атмосфере и гидросфере. Гравитационные, акустические, капилярные и инерционные волны. Волны Россби, Пуанкаре, Кельвина.}
Волновые движения в атмосфере и гидросфере можно классифицировать по типу возвращающей силы на \cite{Носов-2019-13}:
\begin{enumerate}
\item Поверхностные (в случае наличия резкого перепада плотности на границе среды) и внутренние (в случае наличия стратификации внутри среды) гравитационные волны (сила тяжести). Положение равновесия достигается, когда изопикнические линии\footnote{Изопикнические линии -- линии одинаковой плотности.} параллельны изолиниям гравитационного потенциала $\Leftrightarrow$ изопикнические линии перпендикулярны $\vec{g}$.
\item Капилярные волны (сила поверхностного натяжения). Сила поверхностного натяжения стремится вернуть возмущенную поверхность к минимальной площади.
\item Акустические волны (сила упругости). Сила упругости стремится разжать сжимаемую сплошную среду.
\item Инерционные волны (сила Кориолиса).
\end{enumerate}
К инерционным волнам относятся:
\begin{enumerate}
\item Волны Россби\footnote{Волны Россби планетарного масштаба (размера порядка 10000 км) называют волнами Россби-Блиновой.} = теория мелкой вращающейся (в приближении $\beta$-плоскости\footnote{Разложение параметра Кориолиса в ряд Тейлора.}) воды. Дисперсионны: фазовая скорость направлена на запад, групповая скорость не имеет выделенного направления. Механизм образования: при меридиональном переносе из-за наличия (изменения) силы Кориолиса и закона сохранения потенциального вихря (\ref{eq-3-6-1}), высвобождается энергия, которая идёт на генерацию волн.
\item Волны Кельвина = теория мелкой вращающейся (в приближении $f=const$) воды + наличие границы. Недисперсионны: фазовая и групповая скорости совпадают по направлению и равны $\sqrt{gH}$. Представляют собой пограничные волны двух типов: boundary trapped (волны вдоль линии берега или горного хребта) и equatorially trapped (волны вдоль экватора\footnote{Экватор -- граница смена знака силы Кориолиса. Для экваториальных волн Кельвина экватор играет роль стенки, вдоль которой эта волна распространяется в восточном направлении, экспоненциально затухая в сторону высоких широт.}). Каждый тип волн может дополнительно разделяться на поверхностные (баротропные) и внутренние (бароклинные).
\item Волны Пуанкаре = теория мелкой вращающейся (в приближении $f=const$) воды.
\end{enumerate}

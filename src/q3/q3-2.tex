\subsection{Плотностная стратификация и её устойчивость. Адиабатический градиент. Частота Вайсяля-Брента. Внутренние волны.}\label{q3-2}
В зависимости от распределения давления по высоте (плотностной стратификации) среда может иметь:
\begin{enumerate}
\item Нейтральное состояние: $\frac{\partial\rho}{\partial z}=\left(\frac{\partial\rho}{\partial z}\right)_s$, где $\left(\frac{\partial\rho}{\partial z}\right)_s=-\frac{\rho g}{c_s^2}$ -- адиабатический градиент плотности.
\item Устойчивое состояние: $\frac{\partial\rho}{\partial z}<\left(\frac{\partial\rho}{\partial z}\right)_s$
\item Неустойчивое состояние: $\frac{\partial\rho}{\partial z}>\left(\frac{\partial\rho}{\partial z}\right)_s$
\end{enumerate}

Частота малых колебаний устойчиво стратифицированной среды определяется частотой Вайсяля-Брента \cite{Nosov2019-5}:
\begin{equation}\label{eq-3-2-1}
N^2=-\frac{g}{\rho}\left(\frac{\partial\rho}{\partial z}-\left(\frac{\partial\rho}{\partial z}\right)_s\right)=-\frac{g}{\rho}\left(\frac{\partial\rho}{\partial z}+\frac{\rho g}{c_s^2}\right)
\end{equation}
Типичные значения $N$ составляют порядка $10^{-4} - 10^{-1}$ Гц.

Волны, распространяющиеся в устойчиво стратифицированной среде, называются внутренними волнами.
Диапазон изменчивости частот внутренних волн ограничен сверху частотой Вайсяля-Брента.

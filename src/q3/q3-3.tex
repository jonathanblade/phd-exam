\subsection{Термогравитационная конвекция. Уравнение Буссинеска.}
Температурная стратификация является причиной возникновения термогравитационной конвекции.
Конвективная неустойчивость возникает в жидкости или газе, который находится в поле силы тяжести и подогревается снизу, либо охлаждается сверху.
Таким образом возникают ячейки циркуляции (например, роликовая конвекция или конвективные гексагональные ячейки Бинара) \cite{Nosov2019-5}.

Согласно приближению Буссинеска, термогравитационная конвекция обусловлена только вариациями температуры $\widetilde{T}$ (т.е. пренебрегается изменением плотности под влиянием изменения давления), которые приводят к появлению сил плавучести \cite{Носов-2013}.
Полагается, что:
\begin{equation}\label{eq-3-3-1}
\begin{gathered}
T=T_0+\widetilde{T},\quad\widetilde{T}\ll T_0
\\
\rho=\rho_0+\widetilde{\rho},\quad\widetilde{\rho}\ll\rho_0
\\
p=p_0+\widetilde{p},\quad\widetilde{p}\ll p_0
\end{gathered}
\end{equation}
где  $\rho_0$ и $p_0$ описывают основное (среднее) гидростатическое состояние и удовлетворяют уравнению гидростатики (\ref{eq-2-3-1}), а $\widetilde{\rho}$ и $\widetilde{p}$ -- малые отклонения от него.

Для большинства жидких сред выполняется соотношение $\widetilde{\rho}=-\alpha\rho_0\widetilde{T}$, где $\alpha$ -- коэффициент температурного расширения.
С учётом этого конечная система уравнений Буссинеска принимает вид:
\begin{equation}
\begin{gathered}
\underbrace{\frac{\partial \vec{v}}{\partial t}+\left(\vec{v}\cdot\vec{\nabla}\right)\vec{v}=-\frac{\vec{\nabla}\widetilde{p}}{\rho_0}-\vec{b}+\nu\Delta\vec{v}}_\text{уравнение движения (Навье-Стокса)}
\\
\underbrace{\frac{\partial \widetilde{T}}{\partial t}+\left(\vec{v}\cdot\vec{\nabla}\right)\widetilde{T}=\chi\Delta\widetilde{T}}_\text{уравнение переноса тепла (теплопроводности)}
\\
\underbrace{\vec{\nabla}\cdot\vec{v}=0}_\text{уравнение неразрывности (несжимаемости)}
\end{gathered}
\end{equation}
где $\vec{b}=\alpha\widetilde{T}\vec{g}$ -- вектор плавучести, $\chi$ -- коэффициент молекулярной температуропроводности.

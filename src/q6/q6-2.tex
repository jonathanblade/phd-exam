\subsection{Электрические заряды в облаках. Механизмы разделения зарядов в облаках. Молнии. Грозы.}
Механизмы \textit{микроэлектризации} (образования заряда) облаков \cite{Кашлева-2008}:
\begin{enumerate}
\item Ионная электризация, которая поддерживается излучением радиоактивных веществ с земной поверхности и космическими лучами.
\item Электризация из-за столкновения гидрометеоров\footnote{Гидрометеоры -- водяные капли, льдинки и снежная крупа.}.
\end{enumerate}
Механизмы \textit{макроэлектризации} (разделения заряда) облаков \cite{Кашлева-2008}:
\begin{enumerate}
\item Выпадение осадков (из-за захвата ионов гидрометеорами). Макроразделение зарядов в облаке может произойти под влиянием силы тяжести, если заряды разных знаков связаны с гидрометеорами разных размеров и, следовательно, разной массы.
\item Конвективный механизм (из-за захвата конвективным потоком положительных ионов с поверхности Земли). У поверхности Земли преобладают положительно заряженные лёгкие ионы. Это связано с тем, что направленное вниз электрическое поле хорошей погоды побуждает отрицательно заряженные лёгкие ионы подниматься вверх навстречу к положительным ионам, поступающим из космоса, чтобы нейтрализовать их. По мере нарастания конвекции восходящие течения воздушных масс уносят с собой преобладающие вблизи Земли положительные заряды в новообразованную конвективную ячейку.
\end{enumerate}

Из-за дипольной структуры молекулы воды тяжёлые гидрометеоры, летящие вниз, собирают отрицательный заряд, а лёгки гидрометеоры, летящие вверх -- положительный. Итог -- разделение заряда в облаке (снизу отрицательный, а сверху положительный), словно в плоском конденсаторе.
В свою очередь, это создаёт сильное вертикальное электрическое поле внутри облака.
Однако этого поля всё ещё недостаточно, чтобы пробить воздух в любом месте.
Тут большую роль начинают играть высокоскоростные частицы галактических лучей.
В момент пробития грозовое облако пронизывает молния.

Стадии развития молнии:
\begin{enumerate}
\item Распространение из облака отрицательных стримеров (слабо проводящих каналов, заполненных ионизированным газом) и ступенчатых отрицательных лидеров (сильно проводящих каналов, заполненных ионизированным газом). Когда отрицательный стример, растущий из облака вниз, подбирается довольно близко к земле, навстречу ему со стороны земли начинают расти ответные положительные стримеры и лидеры.
\item Встреча отрицательного и положительного стримера (образование стримерной зоны).
\item Встреча отрицательного и положительного лидера (разряд молнии).
\end{enumerate}

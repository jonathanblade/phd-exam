\subsection{Проводимость воздуха. Ионообразование в атмосфере Земли.}
Благодаря присутствию ионов атмосфера не является идеальным изолятором, а обладает способностью проводить электричество.

Ионы в атмосфере образуются в результате процесса ионизации газов, входящих в состав воздуха, когда под воздействием внешнего агента -- ионизатора\footnote{Главнейшими ионизаторами для нижних слоёв атмосферы являются излучения радиоактивных веществ, содержащихся в земной коре и атмосфере, а также космические лучи. Ультрафиолетовые лучи Солнца в области длин волн более 0.2 мкм никакой роли в ионизации этих слоёв атмосферы не играют.} -- молекуле или атому газа сообщается энергия, достаточная для того, чтобы удалить один из наружных валентных электронов атома из сферы действия ядра.
В результате этого первоначально электрически нейтральный атом, в котором положительный заряд ядер равен общему заряду валентных (наружных) электронов, лишившись одного из них, становится положительно заряженным.
Выделившийся же электрон в условиях нормального давления почти мгновенно (за время меньшее $10^{-6}$ с) присоединяется к одном у из нейтральных атомов окружающей среды на дозволенном высшем энергетическом уровне и образует отрицательный ион.

Указанным путём образуются попарно положительный и отрицательный ионы.
Однако такие первично образовавшиеся молекулярные ионы существуют очень недолго (доли секунды), так как под действием поляризационных сил к ним присоединяется некоторое число молекул из окружающего воздуха, в результате чего образуются достаточно устойчивые комплексы молекул, получившие название \textit{лёгких ионов} (нормальных ионов) \cite{Кашлева-2008}.
Но в атмосфере постоянно находятся во взвешенном состоянии посторонние мельчайшие частицы больших размеров (ядра конденсации и другие частицы аэрозоля).
Лёгкие ионы, присоединяясь к ним, отдают им свой заряд.
В результате образуются ионы, имеющие более крупные размеры, так называемые \textit{тяжёлые ионы} (ионы Ланжевена) \cite{Кашлева-2008}.
Также ионы разных знаков могут между собой \textit{рекомбинировать}.

Основной величиной, характеризующей ионизационное состояние атмосферы, является \textit{концентрация ионов}\footnote{В нижних слоях атмосферы концентрация ионов имеет величину порядка $10^{-8}\,\text{м}^{-3}$.}, т.е. число ионов, содержащихся в единице объёма.

Наличие ионов в атмосфере определяет её проводящую способность, или \textit{проводимость}\footnote{В нижних слоях атмосферы проводимость имеет величину порядка $10^{-14}\,\text{Ом}^{-1}\text{м}^{-1}$.} \cite{Кашлева-2008}:
\begin{equation}
\lambda=\sum_{j=1}^\infty\left(n_{-}k_{-}+n_{+}k_{+}\right)_{j}e
\end{equation}
где $n$ -- концентрация ионов, $k$ -- подвижность ионов (в зависимости от размера), $e$ -- заряд электрона.
Расчёты показывают, что проводимость атмосферы более чем на 95 \% обусловлена лёгкими ионами.

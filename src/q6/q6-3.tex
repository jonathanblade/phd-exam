\subsection{Атмосферное электричество и метеорологические процессы. Глобальная электрическая цепь.}
Модель глобальной электрической цепи (ГЭЦ): система Земля–ионосфера, как сферический конденсатор, где поверхность Земли -- отрицательная обкладка, а ионосфера -- положительная.

Основным параметром, характеризующим состояние ГЭЦ является ионосферный потенциал (ИП), который определяется балансом токов зарядки от генераторов ГЭЦ и токов разрядки в областях \textquote{хорошей погоды}\footnote{Области Земли, где отсутствуют облака, ветра, метели. В таких областях электрическое поле у поверхности Земли имеет напряженность порядка $10^2$ В/м и плотностью тока порядка $10^{-12}$ А/$\text{м}^2$ \cite{Морозов-2011}.}.
Его величина составляет 250 -- 300 кВ.

Генераторы в ГЭЦ:
\begin{enumerate}
\item Внутренние. Формируются изменчивостью (унитарной, сезонной, годовой) ИП, который напрямую зависит от количества грозовых облаков в атмосфере.
\item Внешние. Формируются космическим воздействием: изменение ИП из-за флуктуации межпланетного магнитного поля (ММП) или ионизирующего космического излучения.
\end{enumerate}

\subsection{Климатическая изменчивость, квазициклические процессы в земной климатической системе.}
\textit{Эль-Ниньо} -- это квазирегулярное (интервал 2 -- 7 лет) явление, которое характеризуется аномальным увеличением температуры поверхности приэкваториальной области Тихого океана.
Противоположная фаза называется \textit{Ла-Нинья}.

\textit{Южное колебание} -- это сопутствующие атмосферное явление, которое является откликом на Эль-Ниньо (ослабление пассатов) и Ла-Нинья (усиление пассатов).

\textit{Квазидвухлетняя цикличность} -- это явление смены направления зонального ветра в области экваториальной атмосферы (один год -- восточный перенос, другой год -- западный перенос). Средняя продолжительность 26 месяцев.

\textit{Арктическая осцилляция} -- это явление изменения атмосферного давления над арктическим регионом в противофазе с давлением над средними широтами на временных масштабах от недель до десятилетий.

\textit{Северо-Атлантическое колебание} -- это явление перераспределения воздушных масс между Арктикой и субтропической Атлантикой на масштабах десятилетий, которое обусловлено градиентом давления между минимумом Исландского циклона и максимумом Азорского антициклона.
Оказывает наибольшее влияние на зимний климат Северо-Атлантического и Европейского региона.

\textit{Северо-Тихоокеанское колебание} -- это явление перераспределения воздушных масс между Арктикой и субтропической областью Тихого океана на масштабах десятилетий, которое обусловлено градиентом давления между минимумом Алеутского циклона и максимумом Гавайского антициклона.
Оказывает наибольшее влияние на зимний климат Северной Америки.

\subsection{Биогеохимические циклы в земной климатической системе. Модели земной системы.}
\textit{Биогеохимические циклы} -- это циклы, которые связаны с процессами обмена химических веществ между атмосферой, океаном и земными экосистемами.
К ним относятся \cite{Елисеев-2017}:
\begin{enumerate}
\item Углеродный цикл (цикл углекислого газа, метана и других углеродосодержащих соединений). Неорганическая ветвь -- это процессы, связанные с переносом углеродосодержащих соединений. Органическая ветвь -- это процессы, связанные с поглощением углеродосодержащих соединений земными экосистемами. Основными естественными источниками $CO_2$ в атмосфере являются выветривание минералов и вулканическая дегазация. Российские климатические модели с блоком углеродного цикла (модели земной системы): INMCM (ИВМ РАН) и КМ ИФА РАН.
\item Азотный цикл.
\item Фосфорный цикл.
\item Железный цикл.
\end{enumerate}

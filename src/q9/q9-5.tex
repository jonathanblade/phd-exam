\subsection{Климатические модели -- от энергобалансовых до моделей общей циркуляции.}
По степени сложности и исторически по степени совершенства модели климата можно разделить на пять основных классов \cite{Лобанов-2018}:
\begin{enumerate}
\item Энергобалансовые модели (ЭБМ). Основаны на уравнении баланса (законе сохранения) энергии. Моделируют распределение температуры по широте.
\item Радиационно-конвективные модели (РКМ). Основаны на уравнении конвективного теплообмена. Моделируют распределение температуры по высоте.
\item Совместные модели, объединяющие ЭБМ и РКМ. Моделируют распределение температуры по широте и высоте.
\item Модели промежуточной сложности. Предназначены для решения задач, которые не могут быть решены другими моделями, например, моделирование палеоклимата.
\item Модели общей циркуляции атмосферы и океана. Основаны на системе уравнений термо-гидродинамики.
\end{enumerate}

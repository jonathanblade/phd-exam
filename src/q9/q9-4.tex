\subsection{Палеоклиматы. Циклы Миланковича.}
\textit{Циклы Миланковича} -- это циклы, которые описывают изменения климата Земли на промежутках от десяти до ста тысяч лет (глобальные циклы оледенения и потепления) в результате изменения инсоляции\footnote{Инсоляция -- средняя освещённость поверхности солнечном светом.}.

Изменение инсоляции Земли обусловлено тремя эффектами: прецессией (движением земной оси по круговому конусу), нутацией (незначительным колебанием угла наклона земной оси) и изменением формы орбиты.

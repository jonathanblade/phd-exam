\subsection{Общая структура климатической системы.}
Земная климатическая система (ЗКС) включает в себя: атмосферу, гидросферу (океан), литосферу (деятельный слой суши), криосферу (ледники) и биосферу.

\textit{Климат} -- это статистический ансамбль состояний компонент ЗКС за достаточно продолжительный интервал времени (в частности, за 30 лет).

Источником энергии в ЗКС является солнечная радиация.
Однако естественные вариации солнечной постоянной оказываются слабым климатообразующим фактором.
Определяющим (интегрально максимальным) фактором являются парниковые газы.

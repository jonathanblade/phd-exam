\subsection{Параметры чувствительности земной климатической системы к внешним воздействиям. Климатические обратные связи.}
\textit{Климатическая обратная связь} -- это взаимодействие, при котором возмущение одного из количественных показателей климата вызывает изменения в другом показателе, а изменение во втором количественном показателе в конечном итоге ведёт к дополнительному изменению в первом показателе.

Положительные обратные связи усиливают внешнее и антропогенное воздействие. Пример: температура воздуха повышается $\rightarrow$ повышается испарение с поверхности мирового океана $\rightarrow$ увеличивается содержание водяного пара в атмосфере $\rightarrow$ усиливается парниковый эффект $\rightarrow$ температура воздуха повышается.

Отрицательные обратные связи смягчают внешнее и антропогенное воздействие. Пример: температура воздуха повышается $\rightarrow$ повышается испарение с поверхности мирового океана $\rightarrow$ увеличивается облакообразование $\rightarrow$ увеличивается количество отражённой солнечной энергии $\rightarrow$ температура воздуха понижается.

\textit{Чувствительность климата} -- это характеристика, используемая для оценки реакции глобальной климатической системы на заданное внешнее воздействие.
Определяется обратными связями.
Наиболее часто в качестве такого параметра используется \textit{равновесная чувствительность} -- равновесное изменение средней за год глобальной приземной температуры воздуха при удвоении концентрации $CO_2$ в атмосфере.

\subsection{Изотермический, адиабатический и влажно-адиабатический процессы в атмосфере.}
\textit{Адиабатическими процесс} -- термодинамический процесс, происходящий без обмена теплом с окружающей средой.
Атмосферные процессы не являются в полной мере адиабатическими.
Однако, если они происходят достаточно быстро, то влиянием теплообмена можно пренебречь.

В атмосфере при поднятии воздуха вверх происходит адиабатическое расширение (падение давления и температуры), а при опускании -- сжатие (рост давления и температуры).
\textit{Сухадиабатический градиент} $\gamma_a$ температуры равняется $1^\circ$ на 100 м.
Полученное значение относится к сухому, а также к влажному, но ненасыщенному воздуху.
\textit{Влажно-адиабатическим градиентом} $\gamma_b<\gamma_a$ называется градиент температуры во влажном насыщенном воздухе.
Устойчивое состояние атмосферы наблюдается при условии: $\gamma<\gamma_a$ или $\gamma<\gamma_b$.

В атмосфере также наблюдаются случаи, когда температура воздуха с высотой не падает, а:
\begin{itemize}
\item растёт -- \textit{температурная инверсия}. Особенно часто ночью в приземном слое.
\item не изменяется -- \textit{изотермия}.
\end{itemize}

\subsection{Атмосфера и океан как термодинамическая система. Поглощение и рассеяние радиации атмосферными газами и примесями. Парниковый эффект.}\label{q5-1}
Источником энергии в системе атмосфера-океан служит солнечная радиация в диапазоне 0.2--4 мкм.
Средний поток солнечной энергии на расстоянии 1 а.е. от Солнца называется \textit{солнечной постоянной} $J_\oplus$ и имеет величину:
\begin{equation}
J_\oplus=1376\:\text{Вт}/\text{м}^2
\end{equation}

Однако не вся эта энергия доходит до поверхности Земли, часть её отражается или рассеивается в атмосфере.
Отношение потерянной солнечной энергии к падающей называется \textit{альбедо}.
Для Земли альбедо имеет величину порядка 0.3.
В связи с этим средний поток энергии, получаемый единицей поверхности Земли в единицу времени, составляет $240\:\text{Вт}/\text{м}^2$, а не $\frac{\pi R_\oplus^2 J_\oplus}{4\pi R_\oplus^2}=344\:\text{Вт}/\text{м}^2$.

Поверхность Земли, поглощая солнечную радиацию, сама начинает её переизлучать по закону Стефана-Больцмана $E=\sigma T^2$ в диапазоне больших длин волн (в ИК диапазоне), который уже не является прозрачным для атмосферы.
Итог -- поглощение и дальнейшее переизлучение радиации атмосферой (парниковыми газами) или \textit{парниковый эффект}\footnote{Средняя температура на поверхности Земли $+14^\circ$C. Без парникового эффекта (атмосферы) температура Земли была бы равна $-25^\circ$C. Т.е. парниковый эффект нагревает Землю на $39^\circ$.}.
Основными парниковыми газами в атмосфере Земли являются: водяной пар, углекислый газ, метан и азот.

Океан и атмосфера Земли неразрывно связаны друг с другом в рамках сложного процесса.
Ветра, которые дуют над поверхностью океана, передают импульс и механическую энергию воде, порождая тем самым волны и течения.
Океан отдаёт энергию в виде тепла, что является одним из основных источников энергии для циркуляции атмосферы.
Тепло также передается из атмосферы океану, вызывая повышение температуры вод.
Кроме того, между океаном и атмосферой происходит циркуляция газов: главным образом, океан поглощает атмосферный углекислый газ и отдаёт в атмосферу кислород.
Соответственно, вследствие повышения концентрации углекислого газа в атмосфере и связанных с этим атмосферных изменений происходят существенные перемены в основных характеристиках океана.

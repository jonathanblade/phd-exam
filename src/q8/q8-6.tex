\subsection{Солнечный ветер. Полярные сияния. Свечение ночного неба, механизмы возбуждения эмиссий. Солнечно-земные связи и космическая погода.}
\textit{Космическая погода} -- это изменения, вызванные вспышками на Солнце.
Они могут влиять на спутниковые и другие технологии на Земле, а также на жизнь и здоровье людей.
\textquote{Погодой} всё это назвали не случайно.
По словам специалистов Федеральной службы по гидрометеорологии и мониторингу окружающей среды, \textquote{в околоземном пространстве есть и свои бури, и штормы (магнитные и ионосферные), есть свои облака (серебристые, или мезосферные), есть свой ветер (солнечный) и даже свой дождь (так называют одно из явлений в полярной ионосфере) -- все атрибуты погоды налицо}.

\textit{Солнечный ветер} -- это поток плазмы (заряженных частиц) с поверхности Солнца в межпланетное космическое пространство.

\textit{Свечение ночного неба} -- это оптическое свечение верхних слоёв атмосферы (ионосферы), которое обусловлено \textit{хемолюминесценцией} атомов азота и кислорода в процессе рекомбинации.

\textit{Полярные сияния} -- это оптическое свечение верхних слоёв атмосферы (ионосферы), которое обусловлено излучением возбуждённых атомов азота и кислорода из-за заброса заряженных частиц из магнитосферы.

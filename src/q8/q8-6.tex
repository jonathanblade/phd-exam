\subsection{Солнечный ветер. Полярные сияния. Свечение ночного неба, механизмы возбуждения эмиссий. Солнечно-земные связи и космическая погода.}
\textit{Солнечный ветер} -- это поток плазмы (заряженных частиц) с поверхности Солнца в межпланетное космическое пространство.

\textit{Свечение ночного неба} -- это оптическое свечение верхних слоёв атмосферы (ионосферы), которое обусловлено \textit{хемолюминесценцией} атомов азота и кислорода в процессе рекомбинации.

\textit{Полярные сияния} -- это оптическое свечение верхних слоёв атмосферы (ионосферы), которое обусловлено излучением возбуждённых атомов азота и кислорода из-за заброса заряженных частиц из магнитосферы.

\subsection{Магнитосфера Земли. Ионосферно-магнитосферное взаимодействие. Радиационные пояса. Распространение электромагнитных волн в ионосфере. Критические частоты ионосферных слоёв.}
Магнитосфера образуется в результате взаимодействия магнитного поля Земли с солнечным ветром.
Заряженные частицы солнечного ветра движутся к Земле, но отклоняются магнитным полем, тем самым обтекая Землю.
Возникает обтекаемая, вытянутая в форме снаряда полость, называемая магнитосферой.

В первом приближении магнитное поле Земли можно аппроксимировать магнитным диполем с наклоном на $11^\circ$ относительно оси вращения Земли.
На дневной стороне, т.е. на стороне, обращённой к Солнцу, линии магнитного поля сжимаются солнечным ветром, в то время как на ночной стороне они вытягиваются и образуют \textit{магнитосферный хвост}.
Таким образом, магнитосфера простирается примерно на $10R_\oplus$ на дневной стороне и тянется на расстояние более $200R_\oplus$ на ночной стороне.

Токовый слой, являющийся границей магнитосферы между магнитным полем планеты и солнечным ветром, называется \textit{магнитопаузой}.
Её местоположение определяется балансом между давлением магнитного поля и динамическим давлением солнечного ветра.

Магнитосфера не изолирована от солнечного ветра, через её границу (магнитопаузу) постоянно поступают электромагнитная энергия и вещество.
В магнитопаузе выделяют особые области: \textit{каспы}\footnote{Каспы -- это области, разделяющие дневные и ночные силовые линии магнитного поля.} над северным и южным магнитными полюсами планеты.
Это воронки, через которые заряженные частицы проникают в магнитосферу и спускаются к Земле вдоль линий магнитного поля.

Два \textit{радиационных пояса Ван Аллена} представляют собой концентрические пояса высокоэнергичных электронов и протонов, захваченных магнитным полем.
Внутренний протонный пояс (с протонами энергией десятки МэВ) расположен на высотах $1-2R_\oplus$.
Внешний электронный пояс (с электронами энергией в десятки кэВ) расположен на высотах $4-7R_\oplus$.

Максимальная частота электромагнитной волны, которая отражается от слоя ионосферы при вертикальном падении, называется \textit{критической (плазменной) частотой слоя}:
\begin{equation}
f_\text{кр}=\sqrt{\frac{e^2 n_e^\text{м}}{\pi m_e}}\approx 9\cdot 10^3\sqrt{n_e^\text{м}}
\end{equation}
где $n_e^\text{м}$ -- максимальное значение электронной концентрация слоя [$\text{см}^{-3}$].

\subsection{Озоновый слой и его роль в атмосфере Земли. Особенности пространственной изменчивости атмосферного озона. Естественные и антропогенные процессы, влияющие на озоновый слой. Озоновые дыры.}
Озоновый слой -- слой озона ($O_3$) в стратосфере на высотах  20--30 км.
Единицей измерения толщины озонового слоя является единица Добсона (1 ед. Добсона = 10 мкм).
Роль озона -- поглощение УФ излучения Солнца, которое приводит к нагреванию (температурной инверсии) стратосферы.
Естественное образование озона -- фотодиссоциация $O_2$ (цикл Чепмена).
Естественное разрушение озона -- фотодиссоциация $O_3$.

Антропогенное воздействие на озон: выброс фреонов в тропосферу $\rightarrow$ фреоны устойчивы в тропосфере, поэтому удаляются оттуда путём стока в стратосферу $\rightarrow$ под действием УФ излучения фреоны выделяют атомы хлора и брома $\rightarrow$ хлорный и бромный цикл разрушения озона.

Озоновый слой толщиной менее 220 ед. Добсона (2.2 мм) считается озоновый дырой.
Причины формирования антарктической озоновой дыры: зимой в Антарктиде полярная ночь (т.е. озон не образуется) + формируется Антарктический вихрь, который блокирует поступление воздуха с более низких широт (т.е. озон не восполняется) + из-за низкой температуры образуются перламутровые облака, ледяные кристаллы которых, захватывают катализаторы разрушения озона $\rightarrow$ весной кристаллы таят и катализаторы высвобождаются $\rightarrow$ озон уничтожается.

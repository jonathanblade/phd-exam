\subsection{Теория Колмогорова-Обухова. Спектр турбулентности.}
Объектом теории Колмогорова-Обухова\footnote{Иногда теорию Колмогорова-Обухова называют теорией \textquote{К-41} по году выхода (1941) основных работ.} является развитая мелкомасштабная однородная (инвариантная относительно преобразования сдвига) и изотропная (инвариантная относительно преобразования вращения) турбулентность.
В этом случае процесс замыкания уравнений Рейнольдса (\ref{eq-4-3-3}) оказывается наиболее простым.
Колмогоров дополнил идею структурного подхода Ричардсона предположением, что ни смотря на то, что среднее течение неоднородно и анизотропно, мелкомасштабная турбулентность должна быть однородна и изотропна.
Исходя из этих представлений, Колмогоров сформулировал следующие две гипотезы \cite{Носов-2013}:
\begin{enumerate}
\item Статистические характеристики развитой мелкомасштабной турбулентности полностью определяются двумя размерными параметрами: скоростью диссипации $\varepsilon$ и кинематической вязкостью $\nu$. Тогда из теории размерностей можно получить внутренние масштабы длины и времени турбулентности: $\lambda_0=\left(\frac{\nu^3}{\varepsilon}\right)^\frac{1}{4}$, $\tau_0=\left(\frac{\nu}{\varepsilon}\right)^\frac{1}{2}$.
\item Статистические характеристики развитой турбулентности в инерционном интервале ($\lambda_0\ll\lambda\ll L$ и $\tau_0\ll\tau\ll T$, где $L$ и $T$ -- внешние масштабы турбулентности) определяются только скоростью диссипации $\varepsilon$. Тогда из теории размерностей можно получить \textit{закон Колмогорова-Обухова} для скорости турбулентного потока:\begin{equation}\label{eq-4-5-1}v_\lambda\sim(\varepsilon\lambda)^\frac{1}{3}\end{equation}
\end{enumerate}

В спектральном форме закон Колмогорова-Обухова (\ref{eq-4-5-1}) принимает вид \textquote{закона 5/3}:
\begin{equation}
E(k)\sim\varepsilon^\frac{2}{3}k^{-\frac{5}{3}}
\end{equation}

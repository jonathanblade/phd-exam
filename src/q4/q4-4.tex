\subsection{Полуэмпирические теории турбулентности. Пограничные слои.}
Как было упомянуто в вопросе \ref{q4-3}, одного метода Фридмана-Келлера для замыкания уравнений Рейнольдса недостаточно.
Поэтому для решения этой проблемы (нахождения напряжений Рейнольдса) требуется построение дополнительных теорий турбулентности (полуэмпирических теорий, экспериментальных законов, теорий размерностей).

Полуэмпирические теории турбулентности -- это теории турбулентности, которые наряду со строгими уравнениям гидродинамики используют также некоторые дополнительные связи, найденные чисто эмпирически \cite{Носов-2013}.
\begin{enumerate}
\item Теория Буссинеска. Вводит понятие \textit{турбулентной вязкости}, которая существует наряду с кинематической вязкостью $\nu$ и характеризуется коэффициентом $k$.
\item Теория Прандтля и Тейлора. Вводит понятие \textit{длины пути перемешивания} (Прандтль -- для импульса, Тейлор -- для вихря скорости), которая характеризует масштаб турбулентности.
\item Теория Кармана. Предлагает \textit{гипотезу о локальном кинематическом подобии} поля турбулентных пульсаций скорости. Согласно этой гипотезе, поля турбулентных пульсаций скорости подобны друг другу и отличаются лишь масштабами длины и времени (или длины и скорости).
\end{enumerate}

Экспериментальные законы \cite{Носов-2013}:
\begin{enumerate}
\item \textquote{Закон 2/3}. В развитом ($\text{Re}\gg 1$) турбулентном потоке внутри инерционного интервала средний квадрат разности скоростей в двух точках, находящихся на расстоянии $r$ друг от друга равен $C_{2/3}\left(\varepsilon r\right)^\frac{2}{3}$, где $\varepsilon=\langle\frac{\nu}{2}\left(\frac{\partial u_i}{\partial x_j}+\frac{\partial u_j}{\partial x_i}\right)^2\rangle$ -- скорость диссипации (удельная диссипация энергии). Другой формой этого закона является \textquote{закон 5/3}, согласно которому внутри инерционного интервала плотность распределения кинетической энергии по спектру волновых чисел $k$ имеет вид $C_{5/3}\varepsilon^\frac{2}{3}k^{-\frac{5}{3}}$.
\item Закон конечной диссипации энергии. $\lim_{\nu\to 0} \varepsilon=const>0$.
\end{enumerate}

Теории размерностей \cite{Носов-2013}:
\begin{enumerate}
\item Структурный подход Ричардсона. Рассматривает развитую турбулентность, как каскад вихрей разного масштаба. \textquote{В поток бурлящий бросив взгляд, вихрей увидишь там каскад: меньшой энергию у большего берёт, пока мельчайших вязкость не сотрёт}.
\end{enumerate}

При рассмотрении свойств развитой ($\text{Re}\gg 1$) турбулентности вдали от стенок пренебрегается вязкостью.
В пограничных слоях так делать нельзя.
Там наблюдаются большие градиенты скорости, которые как раз и обусловлены вязкостью.
\textit{Универсальный закон турбулентности вблизи стенки} определяет профиль средней скорости в пограничном слое:
\begin{equation}\label{eq-4-4-1}
\langle u(z)\rangle=u^{*}f(\zeta)
\end{equation}
где $u^*$ -- динамическая скорость или скорость трения (величина, имеющая размерность скорости, которая характеризует силу, действующую со стороны турбулентного потока на единицу площади поверхности, вдоль которой течёт этот поток), $\zeta=\frac{zu^*}{\nu}$.
Согласно Карману, функция $f(\zeta)$ имеет логарифмический вид.

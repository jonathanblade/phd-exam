\subsection{Влияние плотностной стратификации на турбулентность. Число Ричардсона. Масштаб Озмидова.}
Если стратификация устойчива, то она может существовать продолжительное время, препятствуя развитию турбулентности.
Неустойчивая стратификация, напротив, провоцирует развитие турбулентности.

В устойчиво стратифицированной среде турбулентность теряет свою энергию на работу против сил плавучести и её максимальный размер определяется \textit{масштабом Озмидова} \cite{Носов-2013}:
\begin{equation}
L_0=\sqrt{\frac{K}{N^2}}
\end{equation}
где $N$ -- частота Вайсяля-Брента (\ref{eq-3-2-1}), $K$ -- удельная кинетическая энергия турбулентности.

Устойчивая стратификации всегда способствует более быстрой диссипации энергии турбулентности.
Но в зависимости от масштаба турбулентного движения это влияние может быть различным:
\begin{itemize}
\item $L\ll L_0$ -- слабое влияние.
\item $L\le L_0$ -- сильное стабилизирующее влияние.
\item $L>L_0$ -- турбулентность не развивается вовсе.
\end{itemize}

Степенью устойчивости стратифицированной среды к развитию турбулентности является \textit{градиентное число Ричардсона}\footnote{Здесь течение предполагается стационарным и плоскопараллельным: $\langle\vec{v}\rangle=\vec{i}\langle u\rangle$.} \cite{Носов-2013}:
\begin{equation}
\text{Ri}=\frac{N^2}{\left(\frac{d\langle u\rangle}{dz}\right)^2}
\end{equation}

В стационарном плоскопараллельном потоке со сдвигом скорости турбулентность развивается при условии $\text{Ri}<\text{Ri}_\text{cr}=\frac{1}{4}$.

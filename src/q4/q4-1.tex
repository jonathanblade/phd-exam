\subsection{Турбулентные и ламинарные течения. Механизмы генерации турбулентности в атмосфере и гидросфере.}
Все течения в атмосфере и гидросфере делятся на два типа: спокойные и плавные течения, называемые ламинарными, и их противоположность -- так называемые турбулентные течения\footnote{Турбулентным потокам свойственно явление чередования ламинарной и турбулентной форм движения, которое именуется \textit{перемежаемостью} \cite{Носов-2013}.}, при которых гидродинамические и термодинамические характеристики (скорость, температура, давление и т.д.) испытывают хаотические флуктуации и потому крайне нерегулярно изменяются в пространстве и во времени.

Турбулентные движения всегда диссипативны, поэтому они не могут поддерживаться сами по себе, а должны черпать энергию из окружающей среды.
Таким образом, основными механизмами генерации турбулентности являются:
\begin{enumerate}
\item Гидродинамическая неустойчивость (энергия турбулентности извлекается из кинетической энергии сдвиговых течений). Следствие -- обрушение поверхностных и внутренних волн.
\item Конвективная неустойчивость (энергия турбулентности извлекается из потенциальной энергии неравномерно нагретой среды в гравитационном поле).
\end{enumerate}

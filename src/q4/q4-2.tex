\subsection{Устойчивость течений. Сдвиговая и конвективная неустойчивости. Числа Рейнольдса и Рэлея.}
Наиболее известный вид гидродинамикой неустойчивости – \textit{сдвиговая неустойчивость} (неустойчивость тангенциальных разрывов скорости или неустойчивость Кельвина-Гельмгольца), которая реализуется, когда один слой жидкости \textquote{скользит} по другому \cite{Носов-2013}.
В данном случае неустойчивости подвержена поверхность раздела между слоями жидкости, которые движутся с различными скоростями.

Уравнение Навье-Стокса (\ref{eq-2-2-1}) без учёта эффекта вращения и нелинейности вязкости имеет вид:
\begin{equation}\label{eq-4-2-1}
\frac{\partial\vec{v}}{\partial t}+\left(\vec{v}\cdot\vec{\nabla}\right)\vec{v}=\vec{g}-\frac{\vec{\nabla}p}{\rho}+\nu\Delta\vec{v}
\end{equation}

Если в (\ref{eq-4-2-1}) перейти к безразмерным величинам:
\begin{equation}
\begin{gathered}
\vec{v}_{*}=\frac{1}{U}\vec{v}
\\
\vec{g}_{*}=\frac{L}{U^2}\vec{g}
\\
p_{*}=\frac{1}{\rho U^2}p
\\
t_{*}=\frac{U}{L}t
\\
\vec{\nabla}_{*}=\frac{1}{L}\vec{\nabla}
\\
\Delta_{*}=\frac{1}{L^2}\Delta
\end{gathered}
\end{equation}
где $U$ -- характерный масштаб скорости, $L$ -- характерный масштаб длины, то получим:
\begin{equation}
\frac{\partial\vec{v}_{*}}{\partial t}+\left(\vec{v}_{*}\cdot\vec{\nabla}_{*}\right)\vec{v}_{*}=\vec{g}_{*}-\vec{\nabla}_{*}p_{*}+\frac{1}{\text{Re}}\Delta_{*}\vec{v}_{*}
\end{equation}
где Re -- \textit{число Рейнольдса}:
\begin{equation}
\text{Re}=\frac{UL}{\nu}\sim\frac{\left(\vec{v}\cdot\vec{\nabla}\right)\vec{v}}{\nu\Delta\vec{v}}
\end{equation}

Фактически Re представляет собой отношение членов уравнения Навье-Стокса, которые \textquote{работают} на развитие неустойчивости и против неё.
Если Re превышает некоторое критического значение $\text{Re}_\text{cr}$ (т.е. $\text{Re}>\text{Re}_\text{cr}$), то движение является турбулентным, в противном случае -- ламинарным.
Для атмосферы и гидросферы значение $\text{Re}_c$ достигает порядка $10^2-10^3$.

\textit{Конвективной неустойчивостью} (неустойчивостью Рэлея-Бенара) называется неустойчивость сплошной среды, которая находится в поле силы тяжести $\vec{g}$ и пронизывается потоком тепла с компонентой в
направлении, противоположном вектору $\vec{g}$ \cite{Носов-2013}.

Основной характеристикой конвективной неустойчивости является \textit{число Рэлея}:
\begin{equation}
\text{Ra}=\frac{g\alpha\Delta TH^3}{\chi\nu}
\end{equation}
где $\alpha$ -- коэффициент температурного расширения, $\chi$ -- коэффициент молекулярной температуропроводности, $\Delta T$ -- перепад температур на высоте $H$.
Следует подчеркнуть, что развитию конвективной неустойчивости препятствует не только молекулярная вязкость, но и молекулярная температуропроводность.

Условием поддержания конвективного движения является неравенство $\text{Ra}>\text{Ra}_\text{cr}$, где $\text{Ra}_\text{cr}$ -- некоторое критическое значение числа Рэлея.

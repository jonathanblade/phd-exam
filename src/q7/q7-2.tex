\subsection{Электромагнитные волны и перенос излучения в атмосфере. Радиационный теплообмен. Атмосферная и ионосферная рефракция.}
\textit{Электромагнитная волна} -- поперечная волна, которые представляет собой распространяющуюся в вакууме со скоростью света систему взаимоортогональных векторов напряжённости электрического $\vec{E}$ и магнитного полей $\vec{H}$.

В общем случае электромагнитная волна описывается системой уравнений Максвелла:
\begin{numcases}{}
\left(\vec{\nabla}\cdot\vec{D}\right)=4\pi\rho\label{eq-7-2-1}
\\
\left(\vec{\nabla}\cdot\vec{B}\right)=0\label{eq-7-2-2}
\\
\left[\vec{\nabla}\times\vec{E}\right]=-\frac{1}{c}\frac{\partial\vec{B}}{\partial t}\label{eq-7-2-3}
\\
\left[\vec{\nabla}\times\vec{H}\right]=\frac{4\pi}{c}\vec{j}+\frac{1}{c}\frac{\partial\vec{D}}{\partial t}\label{eq-7-2-4}
\end{numcases}
где (\ref{eq-7-2-1}) -- закон Гаусса для электрического поля (электрический заряд является источником электрической индукции), (\ref{eq-7-2-2}) -- закон Гаусса для магнитного поля (не существует магнитных зарядов), (\ref{eq-7-2-3}) -- закон индукции Фарадея (изменение магнитной индукции порождает вихревое электрическое поле), (\ref{eq-7-2-4}) -- теорема о циркуляции магнитного поля (электрический ток и изменение электрической индукции порождает вихревое магнитное поле).

Простейшим представлением электромагнитный волны является плоская монохроматическая электромагнитная волна:
\begin{equation}
\begin{gathered}
\vec{E}(\vec{r}, t)=\vec{E_0}(\vec{r})e^{i\left((\vec{k}\cdot\vec{r})-\omega t+ \varphi_0\right)}
\\
\vec{H}(\vec{r}, t)=\vec{H_0}(\vec{r})e^{i\left((\vec{k}\cdot\vec{r})-\omega t+ \varphi_0\right)}
\end{gathered}
\end{equation}
которая описывается уравнениями Гельмгольца:
\begin{equation}
\begin{cases}
\Delta \vec{E}-k^2\frac{\partial^2\vec{E}}{\partial^2 t}=0
\\
\Delta \vec{H}-k^2\frac{\partial^2\vec{H}}{\partial^2 t}=0
\end{cases}
\end{equation}
где $k=\frac{2\pi}{\lambda}=\frac{2\pi f}{v}=\frac{\omega}{v}$ -- волновое число.

В среде скорость распространения электромагнитной волны меньше скорости света и определяется соотношением $v=\frac{c}{\sqrt{\mu\varepsilon}}=\frac{c}{n}$, где $\varepsilon$ -- диэлектрическая проницаемость среды, $\mu$ -- магнитная проницаемость среды, $n$ -- показатель преломления среды.

Поток энергии электромагнитной волны определяется \textit{вектором Умова-Пойтинга}:
\begin{equation}
\vec{J}=\frac{c}{4\pi}\left[\vec{E}\times\vec{H}\right]
\end{equation}
который переносит объёмную плотность энергии:
\begin{equation}
W=\frac{c}{8\pi}\left(\varepsilon E^2+\mu H^2\right)
\end{equation}

При прохождении через среду интенсивность (поток энергии в единицу телесного угла) электромагнитной волны уменьшается по \textit{закону Бугера}:
\begin{equation}
I(l)=I_0e^{-\alpha l}
\end{equation}
где $\alpha=\alpha(\lambda)$ -- коэффициент ослабления.

Показатель преломления воздуха незначительно отличается от 1:
\begin{equation}
n=1+N\cdot10^{-6}
\end{equation}
где $N$ -- приведённый показатель преломления.

В ионосфере для электромагнитных волн короче 10 м приведённый показатель преломления в большей степени определяется концентрацией заряженных частиц $n_e$:
\begin{equation}
N=-40.3\cdot10^6\frac{n_e}{f^2}
\end{equation}
А в тропосфере -- \textit{формулой Бина-Даттона}:
\begin{equation}
N=\frac{79}{T}\left(p+4800\frac{e}{T}\right)
\end{equation}где $T$ -- температура [К], $p$ -- давление воздуха [мбар], $e$ -- парциальное давление водяного пара [мбар].

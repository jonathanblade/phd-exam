\subsection{Рассеяния Рэллея и Ми. Деполяризация рассеянного излучения, параметры Стокса.}
Теория Рэллея описывает упругое рассеяние света на мелких частицах ($D\ll\lambda$), а теория Ми -- на крупных.

Состояние произвольно поляризованной электромагнитной волны (т.е. процесса изменения векторов напряжённости электрического и магнитного полей) описывается при помощи \textit{параметров Стокса}\footnote{Для неполяризованного солнечного света $Q=U=V=0$.}:
\begin{align}
\begin{split}
I&=\langle E_xE^*_x\rangle+\langle E_yE^*_y\rangle
\\
Q&=\langle E_xE^*_x\rangle-\langle E_yE^*_y\rangle
\\
U&=\langle E_xE^*_y\rangle+\langle E^*_xE_y\rangle
\\
V&=-i\left(\langle E_xE^*_y\rangle+\langle E^*_xE_y\rangle\right)
\end{split}
\end{align}

Деполяризация излучения определяется формулой:
\begin{equation}
d=1-P=1-\frac{\sqrt{Q^2+U^2+V^2}}{I}
\end{equation}
где $P$ -- степень поляризации.

Единичная сфера в пространстве Стокса образует так называемую сферу Пуанкаре, центр которой представляет собой неполяризованное состояние ($P=0$), а каждая точка сферы -- различные полностью поляризованные состояния ($P=1$).

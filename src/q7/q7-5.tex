\subsection{Дистанционное зондирование атмосферы и океана.}\label{q7-5}
Современные методы дистанционного зондирования используют излучения в очень широкой спектральном диапазоне -- от УФ до радиодиапазона.

Принцип дистанционного зондирование строится на решении обратной задачи атмосферной оптики, т.е. по известным характеристикам излучения определяются параметры среды.

Большинство обратных задач атмосферной оптики с математической точки зрения сводятся к решению (нахождению ядра) интегральных уравнений Фредгольма I рода:
\begin{equation}
f(x)=\int^b_a K(x,s)\varphi(s)ds
\end{equation}
